\documentclass{article}

\usepackage[utf8]{inputenc}
\usepackage[margin=1.0in]{geometry}
\usepackage{color}
\usepackage{amsmath,amssymb}
\renewcommand{\ij}{_{i,j}}
\newcommand{\ijj}{_{i,j-1}}
\newcommand{\ijk}{_{i,j-2}}
\newcommand{\ijjj}{_{i,j-2}}
\newcommand{\magn}[1]{\left\vert #1 \right\vert }
\renewcommand{\part}[2]{\frac{\partial #1 }{\partial #2}}
\newcommand{\partbig}[2]{\frac{\partial }{\partial #2}\left( #1 \right)}
\newcommand{\harp}{\overset{\rightharpoonup}}
\newcommand{\harpoon}{\overset{\rightharpoonup}}
\newcommand{\ten}[1]{\underline{\underline{#1}}}
\newcommand{\rij}{\harp r \ij}
\newcommand{\rijj}{\harp r \ijj}
\newcommand{\rijjj}{\harp r \ijjj}
\newcommand{\rijk}{\harp r \ijk}


\begin{document}
\begin{center}
  \textbf{Persistence length interactions and force Derviation.}\\
  \textsc{Christian Tabedzki}
\end{center} 



For this report, we derive the forces for each particle involved on the polymer backbone to.

\section{Energies}

The energy is defined as 

\begin{equation}
  \label{eqn:Energy_polymer}
  \beta E_{\text{poly}}  =  \frac{l_p}{2} \sum_{i=1}^{n_p} \int_0^L ds \left(\part{\harp u_i}{s}\right)^2
\end{equation}
which when discretized becomes
\begin{equation}
  \label{eqn:Energy_polymer_discrete}
  \beta E_{\text{poly}}  =  \sum_{i=1}^{n_P}  \sum_{j=2}^{N} \left[
  \frac{\varepsilon_b}{2 l_0}  \magn{ \harp u \ij - \harp u \ijj - \eta \harp R\ij ^\bot} ^2 +
   \frac{\varepsilon_{\vert \vert}}{2 l_0} 
   \left( \harp R\ij \cdot \harp u \ijj - l_0 \gamma \right)^2 +
   \frac{\varepsilon _\perp}
   {2l_0} \magn{\harp R\ijj ^\perp}^2
  \right]
\end{equation}
where 
\[
\harp  R\ij  ^\bot = \harp R \ij - \left(\harp R \ij \cdot \harp u \ijj \right) \harp u \ijj 
\]
which represents the component of \(\harp R \ij\) vector perpendicular to \(\harp u \ijj \).
\(\harp u \ijj \) is defined as  
  \[\harp u_{i,j-1} = 
  \part{\harp{r}_i(s)}{s} = 
    \frac{\left( \harp r\ij-\harp r\ijk \right)}{\magn{\harp r\ij - \harp r\ijk}} \]
which means the orietation of \(\harp u_{i,j-1}\) is independent of the position of \(\harp r\ijj\), where 
\[\harp R \ij = \harp r \ij - \harp r \ijj \]

Additionally, The energnetic coefficients of bending $\varepsilon_b$,
compression $\varepsilon_{\vert \vert}$, and shearing $\varepsilon_\perp$, as
well as $\eta$ and $\gamma$ are taken from Koslover and Spokowitz, with
different degres of chain flexibility \(l_0/l_p\).


\section{Force derivatives}

The forces that act on each particle are taken from the deriative of equation
\ref{eqn:Energy_polymer_discrete}, with respect to each particle position.

First, let's break up the energy term into three parts.
\begin{equation}
  \label{eqn:Energy_polymer_broken}
  \beta E_{\text{poly}}  =  \sum_{i=1}^{n_P}  \sum_{j=2}^{N} \left[
  \overbrace{\frac{\varepsilon_b}{2 l_0}  
  \magn{ \harp u \ij - \harp u \ijj - \eta \harp R\ij ^\bot} ^2}^{\text{Part 1}} 
  +
  \overbrace{
   \frac{\varepsilon_{\vert \vert}}{2 l_0} 
   \left( \harp R\ij \cdot \harp u \ijj - l_0 \gamma \right)^2 }
   ^{\text{Part 2}}
   +
   \overbrace{\frac{\varepsilon _\perp}
   {2l_0} \magn{\harp R\ijj ^\perp}^2} ^ {\text{Part 3}}
  \right]
\end{equation}

\subsection{Derivates of common terms}
\subsubsection{Orientation derivatives}

\begin{align*}
  \part{\harp u \ijj }{\harp r \ij} = \frac{\ten{\delta }}{\magn{\harp r \ij - \harp r \ijjj}} - 
  \frac{
    \left(\rij - \rijk \right)\otimes 
  \left(\rij - \rijk \right)
  }{
    \magn{\rij - \rijjj } ^3
    }
\end{align*}


\begin{align*}
  \part{\harp u \ijj }{\harp r \ijj} = \ten{0}
\end{align*}


\begin{align*}
  \part{\harp u \ijj }{\harp r \ijk} =   \frac{-\ten{\delta }}{\magn{\harp r \ij - \harp r \ijjj}} 
  +
  \frac{
    \left(\rij - \rijk \right)\otimes 
  \left(\rij - \rijk \right)
  }{
    \magn{\rij - \rijjj } ^3
    }
\end{align*}



 \subsubsection{Other area}
\begin{align*}
  \label{eqn:Rperpri}
  \part{\harp R\ij ^ \bot}{\harp r\ij}  &=  \ten{I} 
  - \part{\left(\left(\harp R\ij \cdot \harp u \ijj \right) \otimes 
  \harp u \ijj \right)
  }{\harp r\ij}
\end{align*}

\begin{align*}
  \part{\left(\harp R\ij \cdot \harp u \ijj \right)}{\harp r\ij}   
  &= 
  \harp R\ij \cdot \part{\harp u \ijj}{\harp r \ij} + \part{\harp R \ij}{\harp r \ij} \cdot \harp u\ijj
  \\
  &= 
  \harp R\ij \cdot  \frac{\rij - \harp r\ijjj}{\magn{\harp r \ij - \harp r\ijk}}
\end{align*}

% \subsection{Part 1}


% \subsection{Part 3}

% \begin{align*}
%   \partbig{\magn{\harp R\ijj ^ \perp}^2}{\harp \ij} = 
% \end{align*}




\end{document}