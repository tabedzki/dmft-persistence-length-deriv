\documentclass{article}

\usepackage[utf8]{inputenc}
\usepackage[margin=0.8in]{geometry}
\usepackage{color}
\usepackage{clipboard}
\usepackage{amsmath,amssymb}
\renewcommand{\ij}{_{i,j}}
\newcommand{\ijj}{_{i,j-1}}
\newcommand{\ijk}{_{i,j-2}}
\newcommand{\ijjj}{_{i,j-2}}
\newcommand{\magn}[1]{\left\vert #1 \right\vert }
\renewcommand{\part}[2]{\frac{\partial #1 }{\partial #2}}
\newcommand{\partbig}[2]{\frac{\partial }{\partial #2}\left( #1 \right)}
\newcommand{\harp}{\vec{}}
\newcommand{\harpoon}{\overset{\rightharpoonup}}
\newcommand{\ten}[1]{\underline{\underline{#1}}}
\newcommand{\rij}{\vec{r} \ij}
\newcommand{\Rij}{\vec{R} \ij}
\newcommand{\rijj}{\vec{r} \ijj}
\newcommand{\rijjj}{\vec{r} \ijjj}
\newcommand{\rijk}{\vec{r} \ijk}
\newcommand{\uij}{\vec{u} \ij}
\newcommand{\uijj}{\vec{u} \ijj}
\newcommand{\uijjj}{\vec{u} \ijjj}
\newcommand{\uijk}{\vec{u} \ijk}


\begin{document}
\begin{center}
  \textbf{Persistence length interactions and force Derivation.}\\
  \textsc{Christian Tabedzki}
\end{center} 


For this derivation, we derive the forces for each particle involved on the polymer
backbone to. In this report, normal scalar multiplication is denoted either as
$\times$ or by having the terms next to a vector or tensor, dyadic product
is denoted as $\otimes$ and dot product, denoted as $\cdot$, is either the vector-vector dot product or the vector matrix multiplication.

\section{Energies}

The energy is defined as 

\begin{equation}
  \label{eqn:Energy_polymer}
  \beta E_{\text{poly}}  =  \frac{l_p}{2} \sum_{i=1}^{n_p} \int_0^L ds \left(\part{\vec{u}_i}{s}\right)^2
\end{equation}
which when discretized becomes
\begin{equation}
  \label{eqn:Energy_polymer_discrete}
  \beta E_{\text{poly}}  =  \sum_{i=1}^{n_P}  \sum_{j=2}^{N} \left[
  \overbrace{
  \frac{\varepsilon_b}{2 l_0}  \magn{ \vec{u} \ij - \vec{u} \ijj - \eta 
  \vec{R}\ij ^\bot} ^2 
  }^{\text{bending component}}
  +
\overbrace{
  \frac{\varepsilon_{\vert \vert}}{2 l_0} 
  \left( \vec{R}\ij \cdot \vec{u} \ijj - l_0 \gamma \right)^2 
  }^{\text{compression potential}}
  +
  \overbrace{
  \frac{\varepsilon _\perp}
  {2l_0} \magn{\vec{R}\ij ^\perp}^2
  }^{\text{shearing component}}
  \right]
\end{equation}
where 
\[
\vec{R}\ij  ^\bot = \vec{R} \ij - \left(\vec{R} \ij \cdot \vec{u} \ijj \right)
\vec{u} \ijj  = \Rij - \Rij \cdot \left(
 \uijj \otimes \uijj  
\right)
\]
which represents the component of \(\vec{R} \ij\) vector perpendicular to \(\vec{u} \ijj \).
\(\vec{u} \ijj \) is defined as  
  \[
  \vec{u}_{i,j-1} = 
  \part{\vec{{}r}_i(s)}{s} = 
    \frac{\left( \vec{r}\ij-\vec{r}\ijk \right)}{\magn{\vec{r}\ij - \vec{r}\ijk}} \]
except for the first particle on the chain, which will be defined as 
\[
  \vec{u}_{12} = \frac{\vec{r}_{21}}{\magn{\vec{r}_{21}}} = 
  \frac{\vec{r}_2 - \vec{r}_1}{\magn{\vec{r}_2 - \vec{r}_1}}
\]
which means the orientation of \(\vec{u}_{i,j-1}\) is (for the most part)
independent of the position of \(\vec{r}\ijj\). The difference vector between
two adjacent monomers will be defined as  
\[\vec{R} \ij = \vec{r} \ij - \vec{r} \ijj \]

\noindent
Additionally, the energetic coefficients of bending $\varepsilon_b$,
compression $\varepsilon_{\vert \vert}$, and shearing $\varepsilon_\perp$, as
well as $\eta$ and $\gamma$ were originally taken from Koslover and Spokowitz, with
different degrees of chain flexibility \(l_0/l_p\).


\section{Force derivatives}

The forces that act on each particle are taken from the derivative of equation
\ref{eqn:Energy_polymer_discrete}, with respect to each particle position. We
wil break up the energy term into three parts, but it will also be useful to have the fundamental derivatives out of the way so we can just refer to them.

\begin{equation}
  \label{eqn:Energy_polymer_broken}
  \beta E_{\text{poly}}  =  \sum_{i=1}^{n_P}  \sum_{j=2}^{N} \left[
  \overbrace{\frac{\varepsilon_b}{2 l_0}  
  \magn{ \vec{u} \ij - \vec{u} \ijj - \eta \vec{R}\ij ^\bot} ^2}^{\text{Part 1}} 
  +
  \overbrace{
   \frac{\varepsilon_{\vert \vert}}{2 l_0} 
   \left( \vec{R}\ij \cdot \vec{u} \ijj - l_0 \gamma \right)^2 }
   ^{\text{Part 2}}
   +
   \overbrace{\frac{\varepsilon _\perp}
   {2l_0} \magn{\vec{R}\ij ^\perp}^2} ^ {\text{Part 3}}
  \right]
\end{equation}

\subsection{Derivatives of common terms}
\subsubsection{Orientation derivatives}

\begin{align*}
  \part{\vec{u} \ijj }{\vec{r} \ij} 
  &= \frac{\ten{\delta }}{\magn{\vec{r} \ij - \vec{r} \ijjj}} - 
  \frac{
    \left(\rij - \rijk \right)\otimes 
  \left(\rij - \rijk \right)
  }{
    \magn{\rij - \rijjj } ^3
    }
  \\
  &
  =
  \frac{\ten{\delta}}{\magn{\rij - \rijjj}} 
  -
  \frac{\uijj \otimes \uijj}{\magn{\rij - \rijjj}}\\
  &  =
  \Copy{dudij}
  {
  \frac{\ten{\delta} - \uijj \otimes \uijj}{\magn{\rij - \rijjj}}
  }
\end{align*}


\begin{align*}
  \part{\vec{u} \ijj }{\vec{r} \ijj} = 
  \Copy{dudijj}{ \ten{0}}
\end{align*}

\begin{align*}
  \part{\vec{u} \ijj }{\vec{r} \ijk} 
  &=
     \frac{-\ten{\delta }}{\magn{\vec{r} \ij - \vec{r} \ijjj}} 
  +
  \frac{
    \left(\rij - \rijk \right)\otimes 
  \left(\rij - \rijk \right)
  }{
    \magn{\rij - \rijjj } ^3
    }
  \\
  & =
  \frac{\uijj \otimes \uijj}{\magn{\rij - \rijjj}}
  -
  \frac{\ten{\delta}}{\magn{\rij - \rijjj}} 
  \\
  &  =
  \Copy{dudijjj}
  {
  \frac{\uijj \otimes \uijj - \ten{\delta} }{\magn{\rij - \rijjj}}
  }
\end{align*}


\subsubsection{Orientation derivatives (unique case)}

\begin{align*}
  \part{\vec{u}_{12}}{\vec{r}_{2}} 
  &=
  \partbig{\frac{\vec{r}_2 - \vec{r}_1}{\magn{\vec{r}_2 - \vec{r}_1}}}
  {\vec{r}_{2}}
  \\
  &=
  \frac{\ten{\delta}}
  {\magn{\vec{r}_{2} - \vec{r}_{1}}} 
  - 
  \frac{(\vec{r}_{2} - \vec{r}_{1} ) 
  \otimes 
  (\vec{r}_{2} - \vec{r}_{1} )
  }
  {\magn{\vec{r}_{2} - \vec{r}_{1}}} \\
  & = 
  \frac{
  \ten{\delta}  
  -
  (\vec{r}_{2} - \vec{r}_{1} ) 
  \otimes 
  (\vec{r}_{2} - \vec{r}_{1} )
  }
  {\magn{\vec{r}_{2} - \vec{r}_{1}}} 
\end{align*}


The only difference that is going to occur is that instead of using the further
most point, you shift the index so that the orientation is just $\Rij$ forward
or backward. The change is trivial and can be found in the code. These unique
cases are not derived in this report but the logic of the case given can be
applied to these edge cases.



\begin{align*}
  \part{\vec{u}_{12}}{\vec{r}_{1}} 
  &=
  \partbig{\frac{\vec{r}_2 - \vec{r}_1}{\magn{\vec{r}_2 - \vec{r}_1}}}
  {\vec{r}_{1}}
  \\
  &=
  \frac{-\ten{\delta}}
  {\magn{\vec{r}_{2} - \vec{r}_{1}}} 
  + 
  \frac{(\vec{r}_{2} - \vec{r}_{1} ) 
  \otimes 
  (\vec{r}_{2} - \vec{r}_{1} )
  }
  {\magn{\vec{r}_{2} - \vec{r}_{1}}} \\
  & = 
  \frac{
  (\vec{r}_{2} - \vec{r}_{1} ) 
  \otimes 
  (\vec{r}_{2} - \vec{r}_{1} )
  -
  \ten{\delta}  
  }
  {\magn{\vec{r}_{2} - \vec{r}_{1}}} 
\end{align*}

 \subsubsection{Calculating $\displaystyle\partbig{\Rij^\perp}{\rij}$}

% Calculation for the derivative with respect to d$\rij$
\begin{align*}
  \label{eqn:Rperpri}
  \part{\vec{R}\ij ^ \bot}{\vec{r}\ij}  
  &=
  \part{\Rij}{\rij}
  - \part{\left(\left(\vec{r}\ij \cdot \vec{u} \ijj \right) \otimes 
  \vec{u} \ijj \right)
  }{\vec{r}\ij}
  \\
  &=
  \ten{\delta} 
  - \part{\left(\left(\vec{R}\ij \cdot \vec{u} \ijj \right) \otimes 
  \vec{u} \ijj \right)
  }{\vec{r}\ij}
\end{align*}

\begin{align*}
  \part{\left(\vec{R}\ij \cdot \vec{u} \ijj \right)}{\vec{r}\ij}   
  &= 
  \vec{R}\ij \cdot \part{\vec{u} \ijj}{\vec{r} \ij} + \part{\vec{R} \ij}{\vec{r} \ij} \cdot \vec{u}\ijj
  \\
  &= 
  \vec{R}\ij \cdot 
  \left(
  {
  \frac{\ten{\delta} - \uijj \otimes \uijj}{\magn{\rij - \rijjj}}
  }
  \right)
  +
  \part{\vec{R} \ij}{\vec{r} \ij} \cdot \vec{u}\ijj
  \\
  &= 
  \vec{R}\ij \cdot 
  \left(
  {
  \frac{\ten{\delta} - \uijj \otimes \uijj}{\magn{\rij - \rijjj}}
  }
  \right)
  + \ten{\delta} \cdot \vec{u} \ijj 
  \\
  &= 
  \left(
  {
  \frac{\Rij - (\Rij \cdot \uijj ) \times \uijj}{\magn{\rij - \rijjj}}
  }
  \right)
  + 
  \vec{u} \ijj 
  \\ 
  &=
  \frac{\Rij}{\magn{\rij - \rijjj}}
  \left(
   \ten{\delta} - \uijj \otimes \uijj 
  \right)
  + \uijj
\end{align*}

\begin{align*}
  \part{\left(\left(\vec{R}\ij \cdot \vec{u} \ijj \right) 
  % \otimes 
  \vec{u} \ijj \right)
  }{\vec{r}\ij} 
  &= 
  \part{\left(\vec{R}\ij \cdot \vec{u} \ijj \right)}{\vec{r}\ij}  
  \otimes
  \vec{u} \ijj  + 
  \left(\vec{R}\ij \cdot \vec{u} \ijj \right)
  % \otimes 
  \part{\vec{u} \ijj }{\vec{r}\ij}
   \\
  &= 
  \left(
  \left(
  {
  \frac{\Rij - (\Rij \cdot \uijj ) \times \uijj}{\magn{\rij - \rijjj}}
  }
  \right)
  + 
  \vec{u} \ijj 
  \right) 
  \otimes \uijj
  \\
  &
   \quad 
  +  
  \left(\vec{R}\ij \cdot \vec{u} \ijj \right)
  % \otimes 
  % \underbrace{\part{\vec{u} \ijj }{\vec{r}\ij} }_{\text{Derived above}}
  \left(
  {
  \frac{\ten{\delta} - \uijj \otimes \uijj}{\magn{\rij - \rijjj}}
  }
  \right)
  \\
  &=
  \uijj \otimes \uijj
  + 
  \frac{\Rij \otimes \uijj}{\magn{\rij - \rijjj}} 
  +
  \frac{
    \left( 
      \Rij \cdot \uijj   
    \right)
    \ten{\delta}
    }{\magn{\rij - \rijjj}}
  -
  \frac{
    2 \left(
      \Rij \cdot \uijj 
    \right)
    \left(\uijj \otimes \uijj 
    \right)
    }{\magn{\rij - \rijjj}}
    \\
    & = 
    \uijj \otimes \uijj
     + 
     \frac{1}{\magn{\rij - \rijjj}} 
     \left(
      \Rij \otimes \uijj + 
      \left(\Rij \cdot \uijj \right)
      \times (\ten{\delta} - 2 \uijj \otimes \uijj)
     \right)
\end{align*}

Now plugging in that definition into the derivative we're looking for.

\begin{align*}
  \part{\vec{R}\ij ^ \bot}{\vec{r}\ij}  &=  \ten{\delta} 
  - \part{\left(\left(\vec{R}\ij \cdot \vec{u} \ijj \right) \otimes 
  \vec{u} \ijj \right)
  }{\vec{r}\ij} \\
  &= 
  \boxed{
  \ten{\delta} 
  -
  \left[
  \uijj \otimes \uijj
  + 
  \frac{\Rij \otimes \uijj }{\magn{\rij - \rijjj}} 
  +
  \frac{ \left( 
    \Rij \cdot \uijj
    \right) \ten{\delta}}{\magn{\rij - \rijjj}}
  -
  \frac{
    2 \left(
      \Rij \cdot \uijj 
    \right)
    \left(\uijj \otimes \uijj 
    \right)
    }{\magn{\rij - \rijjj}}
  \right]
  }
\end{align*}
 
%% End of the first derivative calculation. 


\subsubsection{Calculating $\displaystyle\partbig{\Rij^\perp}{\rijj}$}
% And now the derivative for d$\rijj$
\begin{align*}
  \part{\vec{R}\ij ^ \bot}{\vec{r}\ijj}  &= - \ten{\delta} 
  - \part{\left(\left(\vec{R}\ij \cdot \vec{u} \ijj \right) \otimes 
  \vec{u} \ijj \right)
  }{\vec{r}\ijj}
\end{align*}


\begin{align*}
  \part{\left(\vec{R}\ij \cdot \vec{u} \ijj \right)}{\vec{r}\ijj}   
  &= 
  \vec{R}\ij \cdot \part{\vec{u} \ijj}{\vec{r} \ijj} + \part{\vec{R} \ij}{\vec{r} \ijj} \cdot \vec{u}\ijj
  \\
  &= 
  \Rij \cdot \left( \ten{0}
  \right) 
  - \ten{\delta} \cdot \frac{\left( \rij - \rijjj \right)}{\magn{\rij - \rijjj}} 
  \\ 
  &= 
  - \ten{\delta} \cdot \frac{\left( \rij - \rijjj \right)}{\magn{\rij - \rijjj}} 
  =
  - \ten{\delta} \cdot \uijj
  \\
  &
  =
  \boxed{
  - \uijj
  }
\end{align*}

\begin{align*}
  \partbig{\left(\vec{R}\ij \cdot \vec{u} \ijj \right) 
  \vec{u} \ijj 
  }{\vec{r}\ijj}
  % \left(\vec{R}\ij \cdot \vec{u} \ijj \right) \otimes 
  % \vec{u} \ijj 
  &= 
  \part{\left(\vec{R}\ij \cdot \vec{u} \ijj \right)}{\vec{r}\ijj} \otimes
  \uijj 
  +
  \left(\vec{R}\ij \cdot \vec{u} \ijj \right)
  \part{\uijj }{\vec{r}\ijj} 
  \\
  &= 
  - \ten{\delta} \cdot \uijj 
  % \left(\frac{\rijjj - \rijj }{\magn{\rijjj - \rijj}} 
  % - (\Rij) \cdot \frac{\uijj \otimes \uijj}{\magn{\rij - \rijjj}}\right)
  \otimes \uijj 
  + \left(\Rij \cdot \uijj \right) \times 0
  \\
  &= 
  - \uijj \otimes \uijj
  % \\
  % &= 
  % \left(\frac{\rijjj - \rijj }{\magn{\rijjj - \rijj}} 
  % - (\Rij) \cdot \frac{\uijj \otimes \uijj}{\magn{\rij - \rijjj}}\right)
  % \otimes \uijj 
  % \\
  % &= 
  % \left(\rijjj - \rijj  
  % - (\Rij) \cdot \uijj \otimes \uijj\right)
  % \otimes \frac{\uijj }{\magn{\rij - \rijjj}}
\end{align*}

Now plugging in that definition into the derivative we're looking for.

\begin{align*}
  \part{\vec{R}\ij ^ \bot}{\vec{r}\ijj}  &= - \ten{\delta} 
  - \part{\left(\left(\vec{R}\ij \cdot \uijj \right) \otimes 
  \vec{u} \ijj \right)
  }{\vec{r}\ijj} 
  \\
  &= 
  -
  \ten{\delta}
  -
  \left( - \uijj \otimes \uijj \right)
  \\
  &= 
  \boxed{
  \uijj \otimes \uijj 
  -
  \ten{\delta} 
  }
\end{align*}


\subsubsection{Calculating $\displaystyle\partbig{\Rij^\perp}{\rijjj}$}
By symmetry, we can figure out the value by adding the two pervious derivations
and putting a negative in front of it. 
\begin{equation}
  \label{eqn:dRijPerpdrijjj}
  \part{\vec{R}\ij ^ \bot}{\vec{r}\ijk}  =  
  \left[
  \frac{\Rij \otimes \uijj }{\magn{\rij - \rijjj}} 
  +
  \frac{ \left( 
    \Rij \cdot \uijj
    \right) \ten{\delta}}{\magn{\rij - \rijjj}}
  -
  \frac{
    2 \left(
      \Rij \cdot \uijj 
    \right)
    \left(\uijj \otimes \uijj 
    \right)
    }{\magn{\rij - \rijjj}}
  \right]
\end{equation}
However, we should double check this above value but let's do the derivation
anyway.
% And now the derivative for d$\rijjj$

\begin{align*}
  \part{\vec{R}\ij ^ \bot}{\vec{r}\ijk}  &=  
  0
  - \part{\left(\left(\vec{R}\ij \cdot \vec{u} \ijj \right) \otimes 
  \vec{u} \ijj \right)
  }{\vec{r}\ijjj}
\end{align*}


\begin{align*}
  \part{\left(\vec{R}\ij \cdot \vec{u} \ijj \right)}{\vec{r}\ijk}   
  &= 
  \vec{R}\ij \cdot \part{\vec{u} \ijj}{\vec{r} \ijjj} + 
  \part{\vec{R} \ij}{\vec{r} \ijjj} \cdot \vec{u}\ijj
  \\
  &= 
  \underbrace{
  \left(
  \Rij \cdot  \part{\uijj}{\rijjj} 
  \right)}_{\text{Defined before}}
  +
  \ten{0} 
  \\
  &=
  \Rij \cdot 
  \left(
    \frac{\uijj \otimes \uijj - \ten{\delta}}{\magn{\rij - \rijjj}}
  \right)
\end{align*}



\begin{align*}
  \left(  
  \part{
  \left(\vec{R}\ij \cdot \vec{u} \ijj \right)\otimes \uijj}{\vec{r}\ijk}  
  \right)
  &= 
  \left(  
  \part{
  \left(\vec{R}\ij \cdot \vec{u} \ijj \right)}{\vec{r}\ijk}  
  \right)
  \otimes \uijj 
  + 
  \vec{R}\ij \cdot \vec{u} \ijj
  \otimes
  \left(  
  \part{
  ( \uijj  )}{\vec{r}\ijk}  
  \right)
  \\
  &= 
  % \left(  
  % \Rij \cdot  \part{\uijj}{\rijjj} 
  % \right)
  % \otimes \uijj 
  \Rij \cdot 
  \left(
    \frac{\uijj \otimes \uijj - \ten{\delta}}{\magn{\rij - \rijjj}}
  \right)
  \otimes \uijj 
  + 
  \left(
  \vec{R}\ij \cdot \vec{u} \ijj
  \right)
  {
  \left(
  \frac{\uijj \otimes \uijj - \ten{\delta} }{\magn{\rij - \rijjj}}
  \right)
  }
  \\
  &= 
  \Copy{drijdrijjj_early}{
  \left(
    \frac{
      \left(
        \Rij\cdot \uijj 
      \right)
      \left[
       \uijj 
      \otimes \uijj 
      \right]
       - 
  \Rij 
  \otimes \uijj 
      }{\magn{\rij - \rijjj}}
  \right)
  + 
  \left(
  \vec{R}\ij \cdot \vec{u} \ijj
  \right)
  {
  \left(
  \frac{\uijj \otimes \uijj - \ten{\delta} }{\magn{\rij - \rijjj}}
  \right)
  }
  }
  \\
  &= 
    \frac{
  2
      \left(
        \Rij\cdot \uijj 
      \right)
      \left[
       \uijj 
      \otimes \uijj 
      \right]
      }{\magn{\rij - \rijjj}}
       - 
    \frac{
  \Rij 
  \otimes \uijj 
      }{\magn{\rij - \rijjj}}
  {
    -
  \frac{
  \left(
    \Rij 
    \cdot 
    \uijj 
  \right)
  \ten{\delta}
    }{\magn{\rij - \rijjj}}
  }
\end{align*}


Now plugging in that definition into the derivative we're looking for.

\begin{align*}
  \part{\vec{R}\ij ^ \bot}{\vec{r}\ijk}  &=  
  \boxed{
    \frac{
      -
  2
      \left(
        \Rij\cdot \uijj 
      \right)
      \left[
       \uijj 
      \otimes \uijj 
      \right]
      }{\magn{\rij - \rijjj}}
       + 
    \frac{
  \Rij 
  \otimes \uijj 
      }{\magn{\rij - \rijjj}}
  {
    +
  \frac{
  \left(
    \Rij 
    \cdot 
    \uijj 
  \right)
  \ten{\delta}
    }{\magn{\rij - \rijjj}}
  }
  }
\end{align*}
which is the same answer that we got as from Eq \ref{eqn:dRijPerpdrijjj}, so our math is
right.

\subsection{Part 1 Derivation}

\begin{equation}
  % \label{<label>}
  \magn{\uij - \uijj - \eta \Rij^\perp}^2 
  =
  \left(\uij - \uijj - \eta \Rij^\perp\right)
  \cdot 
  \left(\uij - \uijj - \eta \Rij^\perp\right)
\end{equation}

\begin{align*}
  \part{\magn{\uij - \uijj - \eta \Rij^\perp}^2 }{
    \vec{r} _{i,j+1}
  }
  &=
  2\left(\uij - \uijj - \eta \Rij^\perp\right) \cdot
  \part{\left(\uij - \uijj - \eta \Rij^\perp\right)}{
    \vec{r} _{i,j+1}
  }
  \\
  &=
  2\left(\uij - \uijj - \eta \Rij^\perp\right) \cdot
  \part{\left(\uij \right)}{
    \vec{r} _{i,j+1}
  }
  \\
  &=
  2\left(\uij - \uijj - \eta \Rij^\perp\right) \cdot
  % \\
  % & 
  % \quad
  \left(
  \frac{\ten{\delta} - \uij \otimes \uij }{\magn{
    \vec{r} _{i,j+1}
    - \rijj}}
  \right)
\end{align*}

\begin{align*}
  \part{\magn{\uij - \uijj - \eta \Rij^\perp}^2 }{
    \rij 
  }
  &=
  2\left(\uij - \uijj - \eta \Rij^\perp\right) \cdot
  \part{\left(\uij - \uijj - \eta \Rij^\perp\right)}{\rij}
  \\
  &=
  2\left(\uij - \uijj - \eta \Rij^\perp\right) \cdot
  \part{\left(- \uijj \right)}{\rij}
  \\
  &=
  -
  2\left(\uij - \uijj - \eta \Rij^\perp\right) \cdot
  \left(
  \part{\left(\uijj \right)}{\rij}
  \right)
  \\
  &=
  -
  2\left(\uij - \uijj - \eta \Rij^\perp\right) \cdot
  \left(
  \part{\left(\uijj \right)}{\rij}
  +
  \eta
  \part{\left( \Rij^\perp\right)}{\rij}
  \right)
  \\
  &=
  -
  2\left(\uij - \uijj - \eta \Rij^\perp\right) \cdot
  \left(
  \frac{\ten{\delta} - \uijj \otimes \uijj}{\magn{\rij - \rijjj}}
  + \right. \\
  & \quad 
  \left. 
  \eta
  \left[
  \ten{\delta} 
  -
  \left[
  \uijj \otimes \uijj
  + 
  \frac{\Rij \otimes \uijj }{\magn{\rij - \rijjj}} 
  % \right. \right. 
  % \right. 
  % \\
  % & 
  % \quad 
  % \left. \left. 
  % \left. 
  +
  \frac{ \left( 
    \Rij \cdot \uijj
    \right) \ten{\delta}}{\magn{\rij - \rijjj}}
  -
  \frac{
    2 \left(
      \Rij \cdot \uijj 
    \right)
    \left(\uijj \otimes \uijj 
    \right)
    }{\magn{\rij - \rijjj}}
  \right]
  \right]
  \right)
\end{align*}


\begin{align*}
  \part{\magn{\uij - \uijj - \eta \Rij^\perp}^2 }{
    \rijj
  }
  &=
  2\left(\uij - \uijj - \eta \Rij^\perp\right) \cdot
  \part{\left(\uij - \uijj - \eta \Rij^\perp\right)}{\rijj}
  \\
  &=
  2\left(\uij - \uijj - \eta \Rij^\perp\right) \cdot
  \part{\left(\uij - \eta \Rij^\perp\right)}{\rijj} 
  \\
  &=
  2\left(\uij - \uijj - \eta \Rij^\perp\right) \cdot
  % \\
  % & 
  % \quad
  \left(
  \frac{\uij \otimes \uij - \ten{\delta} }{\magn{\rij - \rijjj}}
  -
  \eta 
  \left\{
  \uijj \otimes \uijj 
  -
  \ten{\delta} 
  \right\}
  \right)
\end{align*}


\begin{align*}
  \part{\magn{\uij - \uijj - \eta \Rij^\perp}^2 }{
    \rijjj
  }
  &=
  2\left(\uij - \uijj - \eta \Rij^\perp\right) \cdot
  \part{\left(\uij - \uijj - \eta \Rij^\perp\right)}{\rijjj}
  \\
  & =
  2\left(\uij - \uijj - \eta \Rij^\perp\right) \cdot
  \left(
  \part{ \left( - \uijj \right) }{\rijjj}
  -
  \eta 
  \part{\left( \Rij^\perp\right)}{\rijjj}
  \right)
  \\
  & =
  - 2\left(\uij - \uijj - \eta \Rij^\perp\right) 
  \cdot
  \left(
  \frac{\uijj \otimes \uijj - \ten{\delta} }{\magn{\rij - \rijjj}}
  \right. 
  \\
  &
  \quad 
  +
  \eta 
  \left.
  \left[
  \Rij \cdot 
  \left(
    \frac{\uijj \otimes \uijj - \ten{\delta}}{\magn{\rij - \rijjj}}
  \right)
  + 
  \left(
  \vec{R}\ij \cdot \vec{u} \ijj
  \right)
  {
  \left(
  \frac{\uijj \otimes \uijj - \ten{\delta} }{\magn{\rij - \rijjj}}
  \right)
  }
  \right]
  \right)
\end{align*}


\subsection{Part 2 Derivation}

\begin{align*}
  \part{\left(\Rij \cdot \uijj - l_0\gamma\right)^2}{\rij }
  &=
2(\Rij \cdot \uijj - l_0\gamma)
\otimes
\part{\left(\Rij \cdot \uijj\right)}{\rij}
\\
  &=
2(\Rij \cdot \uijj - l_0\gamma)
  \left[
  \left(
  {
  \frac{\Rij - (\Rij \cdot \uijj ) \times \uijj}{\magn{\rij - \rijjj}}
  }
  \right)
  + 
  \vec{u} \ijj 
  \right] 
\end{align*}

\begin{align*}
  \part{\left(\Rij \cdot \uijj - l_0\gamma\right)^2}{\rijj }
  &=
2(\Rij \cdot \uijj - l_0\gamma)
\part{\left(\Rij \cdot \uijj\right)}{\rijj}
\\
& =
2(\Rij \cdot \uijj - l_0\gamma)
\left[
  - \uijj
\right]
\end{align*}

\begin{align*}
  \part{\left(\Rij \cdot \uijj - l_0\gamma\right)^2}{\rijjj }
  &=
2(\Rij \cdot \uijj - l_0\gamma)
\times
\part{\left(\Rij \cdot \uijj\right)}{\rijjj}
\\
& =
2(\Rij \cdot \uijj - l_0\gamma)
\times
\left[
\Rij 
\cdot 
\left(
  \frac{\uijj \otimes \uijj - \ten{\delta}}{\magn{\rij - \rijjj}}
\right)
\right]
\end{align*}

\subsection{Part 3 Derivation}
For the next three definitions, plug in the values that were previously calculated. 

\begin{align*}
  \part{
    \magn{\Rij^\perp}^2
  }{ \rij }
  &= 
  2 \Rij^\perp \cdot \part{\Rij^\perp}{\rij} 
  \\
  &= 
  2
  \left(
  \vec{R} \ij 
  - 
  \left(\vec{R} \ij \cdot \vec{u} \ijj \right) 
  \vec{u} \ijj 
  \right)
  \cdot 
  \part{\Rij^\perp}{\rij}
  \\
  &= 
  2
  \left(
  \vec{R} \ij 
  - 
  \left(\vec{R} \ij \cdot \vec{u} \ijj \right) 
  \vec{u} \ijj 
  \right)
  \\
  &
  \qquad
  \cdot 
  \left\{
  \ten{\delta} 
  -
  \left[
  \uijj \otimes \uijj
  + 
  \frac{\Rij \otimes \uijj }{\magn{\rij - \rijjj}} 
  +
  \frac{ \left( 
    \Rij \cdot \uijj
    \right) \ten{\delta}}{\magn{\rij - \rijjj}}
  -
  \frac{
    2 \left(
      \Rij \cdot \uijj 
    \right)
    \left(\uijj \otimes \uijj 
    \right)
    }{\magn{\rij - \rijjj}}
  \right]
  \right\}
\end{align*}



\begin{align*}
  \part{
    \magn{\Rij^\perp}^2
  }{ \rijj }
  &=
  \part{\left(
    \Rij^\perp \cdot \Rij^\perp 
    \right)
  }{ \rijj }
  \\
  &= 
  2 \Rij^\perp \cdot \part{\Rij^\perp}{\rijj}
  \\
  &= 
  2
  \left(
  \vec{R} \ij 
  - 
  \left(\vec{R} \ij \cdot \vec{u} \ijj \right) 
  \vec{u} \ijj 
  \right)
  \cdot 
  \part{\Rij^\perp}{\rijj}
  \\
  &= 
  2
  \left(
  \vec{R} \ij 
  - 
  \left(\vec{R} \ij \cdot \vec{u} \ijj \right) 
  \vec{u} \ijj 
  \right)
  \cdot
  \left(
  \uijj \otimes \uijj 
  -
  \ten{\delta} 
  \right)
\end{align*}



\begin{align*}
  \part{
    \magn{\Rij^\perp}^2
  }{ \rijjj }
  &= 
  2 \Rij^\perp \cdot \part{\Rij^\perp}{\rijjj}
  \\
  &= 
  2
  \left(
  \vec{R} \ij 
  - 
  \left(\vec{R} \ij \cdot \vec{u} \ijj \right) 
  \vec{u} \ijj 
  \right)
  \cdot 
  \part{\Rij^\perp}{\rijjj}
  \\
  &= 
  2
  \left(
  \vec{R} \ij 
  - 
  \left(\vec{R} \ij \cdot \vec{u} \ijj \right) 
  \vec{u} \ijj 
  \right)
  \cdot 
  \left\{
    \frac{
      -
  2
      \left(
        \Rij\cdot \uijj 
      \right)
      \left[
       \uijj 
      \otimes \uijj 
      \right]
      }{\magn{\rij - \rijjj}}
       + 
    \frac{
  \Rij 
  \otimes \uijj 
      }{\magn{\rij - \rijjj}}
  {
    +
  \frac{
  \left(
    \Rij 
    \cdot 
    \uijj 
  \right)
  \ten{\delta}
    }{\magn{\rij - \rijjj}}
  }
  \right\}
\end{align*}


In this report, I have gone and expanded out the terms. In the code, I use the
condensed version, meaning \(  \part{\Rij^\perp}{\rijjj}\) is calculated once
before all the updates and energy calculations and then that value is used
throughout. This is done to ensure that there are less calculations and hence
less possibilities for error.



\end{document}