\documentclass{article}

\usepackage[utf8]{inputenc}
\usepackage[margin=1.0in]{geometry}
\usepackage{color}
\usepackage{amsmath,amssymb}
\renewcommand{\ij}{_{i,j}}
\newcommand{\ijj}{_{i,j-1}}
\newcommand{\ijk}{_{i,j-2}}
\newcommand{\ijjj}{_{i,j-2}}
\newcommand{\magn}[1]{\left\vert #1 \right\vert }
\renewcommand{\part}[2]{\frac{\partial #1 }{\partial #2}}
\newcommand{\partbig}[2]{\frac{\partial }{\partial #2}\left( #1 \right)}
\newcommand{\harp}{\overset{\rightharpoonup}}
\newcommand{\harpoon}{\overset{\rightharpoonup}}
\newcommand{\ten}[1]{\underline{\underline{#1}}}
\newcommand{\rij}{\harp r \ij}
\newcommand{\Rij}{\harp R \ij}
\newcommand{\rijj}{\harp r \ijj}
\newcommand{\rijjj}{\harp r \ijjj}
\newcommand{\rijk}{\harp r \ijk}
\newcommand{\uij}{\harp u \ij}
\newcommand{\uijj}{\harp u \ijj}
\newcommand{\uijjj}{\harp u \ijjj}
\newcommand{\uijk}{\harp u \ijk}


\begin{document}
\begin{center}
  \textbf{Persistence length interactions and force Derviation.}\\
  \textsc{Christian Tabedzki}
\end{center} 



For this report, we derive the forces for each particle involved on the polymer backbone to.

\section{Energies}

The energy is defined as 

\begin{equation}
  \label{eqn:Energy_polymer}
  \beta E_{\text{poly}}  =  \frac{l_p}{2} \sum_{i=1}^{n_p} \int_0^L ds \left(\part{\harp u_i}{s}\right)^2
\end{equation}
which when discretized becomes
\begin{equation}
  \label{eqn:Energy_polymer_discrete}
  \beta E_{\text{poly}}  =  \sum_{i=1}^{n_P}  \sum_{j=2}^{N} \left[
  \frac{\varepsilon_b}{2 l_0}  \magn{ \harp u \ij - \harp u \ijj - \eta \harp R\ij ^\bot} ^2 +
   \frac{\varepsilon_{\vert \vert}}{2 l_0} 
   \left( \harp R\ij \cdot \harp u \ijj - l_0 \gamma \right)^2 +
   \frac{\varepsilon _\perp}
   {2l_0} \magn{\harp R\ijj ^\perp}^2
  \right]
\end{equation}
where 
\[
\harp  R\ij  ^\bot = \harp R \ij - \left(\harp R \ij \cdot \harp u \ijj \right) \harp u \ijj 
\]
which represents the component of \(\harp R \ij\) vector perpendicular to \(\harp u \ijj \).
\(\harp u \ijj \) is defined as  
  \[\harp u_{i,j-1} = 
  \part{\harp{r}_i(s)}{s} = 
    \frac{\left( \harp r\ij-\harp r\ijk \right)}{\magn{\harp r\ij - \harp r\ijk}} \]
which means the orietation of \(\harp u_{i,j-1}\) is independent of the position of \(\harp r\ijj\), where 
\[\harp R \ij = \harp r \ij - \harp r \ijj \]

Additionally, The energnetic coefficients of bending $\varepsilon_b$,
compression $\varepsilon_{\vert \vert}$, and shearing $\varepsilon_\perp$, as
well as $\eta$ and $\gamma$ are taken from Koslover and Spokowitz, with
different degres of chain flexibility \(l_0/l_p\).


\section{Force derivatives}

The forces that act on each particle are taken from the deriative of equation
\ref{eqn:Energy_polymer_discrete}, with respect to each particle position.

First, let's break up the energy term into three parts.
\begin{equation}
  \label{eqn:Energy_polymer_broken}
  \beta E_{\text{poly}}  =  \sum_{i=1}^{n_P}  \sum_{j=2}^{N} \left[
  \overbrace{\frac{\varepsilon_b}{2 l_0}  
  \magn{ \harp u \ij - \harp u \ijj - \eta \harp R\ij ^\bot} ^2}^{\text{Part 1}} 
  +
  \overbrace{
   \frac{\varepsilon_{\vert \vert}}{2 l_0} 
   \left( \harp R\ij \cdot \harp u \ijj - l_0 \gamma \right)^2 }
   ^{\text{Part 2}}
   +
   \overbrace{\frac{\varepsilon _\perp}
   {2l_0} \magn{\harp R\ijj ^\perp}^2} ^ {\text{Part 3}}
  \right]
\end{equation}

\subsection{Derivates of common terms}
\subsubsection{Orientation derivatives}

\begin{align*}
  \part{\harp u \ijj }{\harp r \ij} = \frac{\ten{\delta }}{\magn{\harp r \ij - \harp r \ijjj}} - 
  \frac{
    \left(\rij - \rijk \right)\otimes 
  \left(\rij - \rijk \right)
  }{
    \magn{\rij - \rijjj } ^3
    }
\end{align*}


\begin{align*}
  \part{\harp u \ijj }{\harp r \ijj} = \ten{0}
\end{align*}


\begin{align*}
  \part{\harp u \ijj }{\harp r \ijk} =   \frac{-\ten{\delta }}{\magn{\harp r \ij - \harp r \ijjj}} 
  +
  \frac{
    \left(\rij - \rijk \right)\otimes 
  \left(\rij - \rijk \right)
  }{
    \magn{\rij - \rijjj } ^3
    }
\end{align*}



 \subsubsection{Other area}

Calculation for the derivative with respect to d$\rij$
\begin{align*}
  \label{eqn:Rperpri}
  \part{\harp R\ij ^ \bot}{\harp r\ij}  &=  \ten{I} 
  - \part{\left(\left(\harp R\ij \cdot \harp u \ijj \right) \otimes 
  \harp u \ijj \right)
  }{\harp r\ij}
\end{align*}

\begin{align*}
  \part{\left(\harp R\ij \cdot \harp u \ijj \right)}{\harp r\ij}   
  &= 
  \harp R\ij \cdot \part{\harp u \ijj}{\harp r \ij} + \part{\harp R \ij}{\harp r \ij} \cdot \harp u\ijj
  \\
  &= 
  - \harp R\ij \cdot  \frac{(\rij - \harp r\ijjj)(\rij - \harp r\ijjj)}
  {\magn{\harp r \ij - \harp r\ijk}^3}
  + \ten{\delta} \cdot \harp u \ijj 
  \\
  &= 
-\frac{\harp R \ij}{\magn{\rij - \rijjj}} \cdot (\uijj \otimes \uijj ) 
+  \uijj
  % + \frac{\rij - \rijjj}{\magn{\rij - \rijjj } }
\end{align*}

\begin{align*}
  \part{\left(\left(\harp R\ij \cdot \harp u \ijj \right) \otimes 
  \harp u \ijj \right)
  }{\harp r\ij} 
  &= 
  \part{\left(\harp R\ij \cdot \harp u \ijj \right)}{\harp r\ij}  
  \otimes \harp u \ijj  + 
  \left(\harp R\ij \cdot \harp u \ijj \right)
  \otimes 
  \part{\harp u \ijj }{\harp r\ij}
   \\
  &= 
  \left(
  -\frac{\harp R \ij}{\magn{\rij - \rijjj}} \cdot (\uijj \otimes \uijj ) 
  +  \uijj
  \right) \otimes \uijj\\
  & \quad 
  +  
  \left(\harp R\ij \cdot \harp u \ijj \right)
  \otimes 
  \underbrace{\part{\harp u \ijj }{\harp r\ij} }_{\text{Derived above}}
\end{align*}
% \frac{\left( \harp r\ij-\harp r\ijk \right)}{\magn{\harp r\ij - \harp r\ijk}} 

% \begin{align*}
%   - \part{\left(\left(\harp R\ij \cdot \harp u \ijj \right) \otimes 
%   \harp u \ijj\right)
%   }{\harp r\ij}
%   % \left(\harp R\ij \cdot \harp u \ijj \right) \otimes 
%   % \harp u \ijj 
%   &= 
%   \part{\left(\harp R\ij \cdot \harp u \ijj \right)}{\harp r\ij} \otimes
%   \uijj 
%   +
%   \left(\harp R\ij \cdot \harp u \ijj \right) \otimes
%   \part{\uijj }{\harp r\ij} 
%   \\
%   &= 
%   \left(\frac{\rijjj - \rijj }{\magn{\rijjj - \rijj}} 
%   - (\Rij) \cdot \frac{\uijj \otimes \uijj}{\magn{\rij - \rijjj}}\right)
%   \otimes \uijj 
%   + \left(\Rij \cdot \uijj \right) \otimes \ten{0}
%   \\
%   &= 
%   \left(\frac{\rijjj - \rijj }{\magn{\rijjj - \rijj}} 
%   - (\Rij) \cdot \frac{\uijj \otimes \uijj}{\magn{\rij - \rijjj}}\right)
%   \otimes \uijj 
%   \\
%   &= 
%   \left(\rijjj - \rijj  
%   - (\Rij) \cdot \uijj \otimes \uijj\right)
%   \otimes \frac{\uijj }{\magn{\rij - \rijjj}}
% \end{align*}

Now plugging in that definition into the derivative we're looking for.

\begin{align*}
  \part{\harp R\ij ^ \bot}{\harp r\ij}  &=  \ten{\delta} 
  - \part{\left(\left(\harp R\ij \cdot \harp u \ijj \right) \otimes 
  \harp u \ijj \right)
  }{\harp r\ij} \\
  &= 
  \ten{\delta} 
  -
  \left[
  \left(
  -\frac{\harp R \ij}{\magn{\rij - \rijjj}} \cdot (\uijj \otimes \uijj ) 
  +  \uijj
  \right) \otimes \uijj
  +  
  \left(\harp R\ij \cdot \harp u \ijj \right)
  \otimes 
  {\part{\harp u \ijj }{\harp r\ij} }
  \right]
\end{align*}
 
%% End of the first derivative calculation. 


And now the derivative for d$\rijj$
\begin{align*}
  \part{\harp R\ij ^ \bot}{\harp r\ijj}  &= - \ten{\delta} 
  - \part{\left(\left(\harp R\ij \cdot \harp u \ijj \right) \otimes 
  \harp u \ijj \right)
  }{\harp r\ijj}
\end{align*}


\begin{align*}
  \part{\left(\harp R\ij \cdot \harp u \ijj \right)}{\harp r\ijj}   
  &= 
  \harp R\ij \cdot \part{\harp u \ijj}{\harp r \ijj} + \part{\harp R \ij}{\harp r \ijj} \cdot \harp u\ijj
  \\
  &= 
  \Rij \cdot \left( \ten{0}
  \right) 
  - \ten{\delta} \cdot \frac{\left( \rij - \rijjj \right)}{\magn{\rij - \rijjj}} 
  \\ 
  &= 
  - \ten{\delta} \cdot \frac{\left( \rij - \rijjj \right)}{\magn{\rij - \rijjj}} 
  =
  - \ten{\delta} \cdot \uijj
  =
  - \uijj
\end{align*}

\begin{align*}
  \part{\left(\left(\harp R\ij \cdot \harp u \ijj \right) \otimes 
  \harp u \ijj \right)
  }{\harp r\ijj}
  % \left(\harp R\ij \cdot \harp u \ijj \right) \otimes 
  % \harp u \ijj 
  &= 
  \part{\left(\harp R\ij \cdot \harp u \ijj \right)}{\harp r\ijj} \otimes
  \uijj 
  +
  \left(\harp R\ij \cdot \harp u \ijj \right) \otimes
  \part{\uijj }{\harp r\ijj} 
  \\
  &= 
  - \ten{\delta} \cdot \uijj 
  % \left(\frac{\rijjj - \rijj }{\magn{\rijjj - \rijj}} 
  % - (\Rij) \cdot \frac{\uijj \otimes \uijj}{\magn{\rij - \rijjj}}\right)
  \otimes \uijj 
  + \left(\Rij \cdot \uijj \right) \otimes \ten{0}
  \\
  &= 
  - \uijj \otimes \uijj
  % \\
  % &= 
  % \left(\frac{\rijjj - \rijj }{\magn{\rijjj - \rijj}} 
  % - (\Rij) \cdot \frac{\uijj \otimes \uijj}{\magn{\rij - \rijjj}}\right)
  % \otimes \uijj 
  % \\
  % &= 
  % \left(\rijjj - \rijj  
  % - (\Rij) \cdot \uijj \otimes \uijj\right)
  % \otimes \frac{\uijj }{\magn{\rij - \rijjj}}
\end{align*}

Now plugging in that definition into the derivative we're looking for.

\begin{align*}
  \part{\harp R\ij ^ \bot}{\harp r\ijj}  &= - \ten{\delta} 
  - \part{\left(\left(\harp R\ij \cdot \uijj \right) \otimes 
  \harp u \ijj \right)
  }{\harp r\ijj} 
  \\
  &= -
  \ten{\delta}
  -
  \left( - \uijj \otimes \uijj \right)
  \\
  &= 
  -
  \ten{\delta} + 
  \uijj \otimes \uijj 
  % \left(\rijjj - \rijj  
  % - (\Rij) \cdot \uijj \otimes \uijj\right)
  % \otimes \frac{\uijj }{\magn{\rij - \rijjj}}
\end{align*}


And now the derivative for d$\rijjj$

\begin{align*}
  \part{\harp R\ij ^ \bot}{\harp r\ijk}  &=  
  0
  - \part{\left(\left(\harp R\ij \cdot \harp u \ijj \right) \otimes 
  \harp u \ijj \right)
  }{\harp r\ijjj}
\end{align*}


\begin{align*}
  \part{\left(\harp R\ij \cdot \harp u \ijj \right)}{\harp r\ijk}   
  &= 
  \harp R\ij \cdot \part{\harp u \ijj}{\harp r \ijjj} + 
  \part{\harp R \ij}{\harp r \ijjj} \cdot \harp u\ijj
  \\
  &= 
  \ten{0} + 
  \underbrace{
  \left(
  \Rij \cdot  \part{\uijj}{\rijjj} 
  \right)}_{\text{Defined above}}
\end{align*}



\begin{align*}
  \left(  
  \part{
  \left(\harp R\ij \cdot \harp u \ijj \right)\otimes \uijj}{\harp r\ijk}  
  \right)
  &= 
  \left(  
  \part{
  \left(\harp R\ij \cdot \harp u \ijj \right)}{\harp r\ijk}  
  \right)\otimes \uijj 
  + 
  \harp R\ij \cdot \harp u \ijj
  \otimes
  \left(  
  \part{
  ( \uijj  )}{\harp r\ijk}  
  \right)
  \\
  &= 
  \left(  
  \Rij \cdot  \part{\uijj}{\rijjj} 
  \right)
  \otimes \uijj 
  + 
  \harp R\ij \cdot \harp u \ijj
  \otimes
  \left(  
  \part{
  ( \uijj  )}{\harp r\ijk}  
  \right)
\end{align*}



Now plugging in that definition into the derivative we're looking for.

\begin{align*}
  \part{\harp R\ij ^ \bot}{\harp r\ijk}  &=  
  -\left[
  \left(  
  \Rij \cdot  \part{\uijj}{\rijjj} 
  \right)
  \otimes \uijj 
  + 
  \harp R\ij \cdot \harp u \ijj
  \otimes
  \left(  
  \part{
  ( \uijj  )}{\harp r\ijk}  
  \right)
  \right]
\end{align*}

\subsection{Part 1 Derivation}

\begin{align*}
  \magn{\uij - \uijj - \eta \Rij^\perp}^2 
  &=
  \left(\uij - \uijj - \eta \Rij^\perp\right)
  \cdot 
  \left(\uij - \uijj - \eta \Rij^\perp\right)
  \\
  \part{\magn{\uij - \uijj - \eta \Rij^\perp}^2 }{
    \rij 
  }
  &=
  2\left(\uij - \uijj - \eta \Rij^\perp\right) \cdot
  \part{\left(\uij - \uijj - \eta \Rij^\perp\right)}{\rij}
  \\
  &=
  2\left(\uij - \uijj - \eta \Rij^\perp\right) \cdot
  \part{\left(- \uijj - \eta \Rij^\perp\right)}{\rij}
\end{align*}


\begin{align*}
  \part{\magn{\uij - \uijj - \eta \Rij^\perp}^2 }{
    \rijj
  }
  &=
  2\left(\uij - \uijj - \eta \Rij^\perp\right) \cdot
  \part{\left(\uij - \uijj - \eta \Rij^\perp\right)}{\rijj}
  \\
  &=
  2\left(\uij - \uijj - \eta \Rij^\perp\right) \cdot
  \part{\left(\uij - \eta \Rij^\perp\right)}{\rijj}
\end{align*}


\begin{align*}
  \part{\magn{\uij - \uijj - \eta \Rij^\perp}^2 }{
    \rijjj
  }
  &=
  2\left(\uij - \uijj - \eta \Rij^\perp\right) \cdot
  \part{\left(\uij - \uijj - \eta \Rij^\perp\right)}{\rijjj}
\end{align*}


\subsection{Part 2 Derivation}

\begin{align*}
  \part{\left(\Rij \cdot \uijj - l_0\gamma\right)^2}{\rij }
  &=
2(\Rij \cdot \uijj - l_0\gamma)
\cdot 
\part{\left(\Rij \cdot \uijj\right)}{\rij}
\end{align*}

\begin{align*}
  \part{\left(\Rij \cdot \uijj - l_0\gamma\right)^2}{\rijj }
  &=
2(\Rij \cdot \uijj - l_0\gamma)
\cdot 
\part{\left(\Rij \cdot \uijj\right)}{\rijj}
\end{align*}

\begin{align*}
  \part{\left(\Rij \cdot \uijj - l_0\gamma\right)^2}{\rijjj }
  &=
2(\Rij \cdot \uijj - l_0\gamma)
\cdot 
\part{\left(\Rij \cdot \uijj\right)}{\rijjj}
\end{align*}

\subsection{Part 3 Derivation}
For the next three definitions, plug in the values that were previously calculated. 

\begin{align*}
  \part{
    \magn{\Rij^\perp}^2
  }{ \rij }
  &= 
  2 \Rij^\perp \cdot \part{\Rij^\perp}{\rij}
\end{align*}
\begin{align*}
  \part{
    \magn{\Rij^\perp}^2
  }{ \rijj }
  &= 
  2 \Rij^\perp \cdot \part{\Rij^\perp}{\rijj}
\end{align*}
\begin{align*}
  \part{
    \magn{\Rij^\perp}^2
  }{ \rijjj }
  &= 
  2 \Rij^\perp \cdot \part{\Rij^\perp}{\rijjj}
\end{align*}
% \subsection{Part 1}


% \subsection{Part 3}

% \begin{align*}
%   \partbig{\magn{\harp R\ijj ^ \perp}^2}{\harp \ij} = 
% \end{align*}




\end{document}